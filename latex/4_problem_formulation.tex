Consider the system $\dot{x}=f(x, u, d)$ with $x\in\mathbb{R}^n$, $u\in U\subseteq \mathbb{R}^m$, $d\in D\subseteq \mathbb{R}^p$, $f:\mathbb{R}^n \times U \times D \rightarrow \mathbb{R}^n$ and $t \in [\tau_i, \tau_f]$. The inputs $u(\cdot)$, $d(\cdot)$ represent Player 1 (control) and Player 2 (disturbance) respectively, we assume that they are drawn from the set of measurable functions: (WHY??)
\[ 
    u(\cdot) \in \mathcal{U}_{[\tau_i, \tau_f]}  \triangleq 
    \left\{
        \sigma: [\tau_i, \tau_f] \rightarrow U| \sigma(\cdot) \, \textrm{is measurable} \, 
    \right\} 
\]
\[ 
    d(\cdot) \in \mathcal{D}_{[\tau_i, \tau_f]}  \triangleq 
    \left\{
        \sigma: [\tau_i, \tau_f] \rightarrow D| \sigma(\cdot) \, \textrm{is measurable} \, 
    \right\} 
\]

Consider also two functions $g: \mathbb{R}^n \rightarrow \mathbb{R}$, $h: \mathbb{R}^n \rightarrow \mathbb{R}$ used to represent the target set $R$ and the avoid set $A$ respectively (\ref{g}) (\ref{h}). Assume $U$, $D$ are compact, $f(\cdot)$, $g(\cdot)$, $h(\cdot)$ are bounded and  Lipschitz continuous in $x$ and continuous in $u$ and $d$, therefore, the system dynamics admits a unique trajectory $x(t)$ from an initial state $x_i$ at time $\tau_i$ under input $u(\cdot)$ and $d(\cdot)$. We denote this solutions as:

\[
    \phi(t; x_i, \tau_i, u(\cdot), d(\cdot)) : [\tau_i, \tau_f] \rightarrow \mathbb{R}^n
\]

In order to define the information patterns of the differential game is necessary to compute the RAS, we assume that the control input $u$ (Player 1), is restricted to use a non-anticipative strategies $\gamma(\cdot)$ as before.
Given the previous sets $R$ and $A$ (\ref{R_A}), whereby for technical reasons we assume $R$ closed and $A$ open, we can define two different kinds of reachability problems. In the first one we are interested in reaching safely $R$ exactly at the end of the game $(t=\tau_f)$, in the second instead, the system can reach safely $R$ at any $t$ inside the time horizon $[\tau_i, \tau_f]$. In the following we will formulate and solve both of them.

\subsection{Reach-avoid at the terminal time}
In this first type of reachability problem, we are interested in characterizing the RAS as the set of initial states from which the system trajectory $\sigma(\cdot)$ can start and reach the target set $R$ at the terminal time $\tau_f$, without passing through the avoid set $A$ over the time interval $[\tau_i, \tau_f]$. Formally the RAS contains all the initial states $x_i$ for which there exists an optimal strategy $\gamma[d](t) \in \Gamma{[\tau_i, \tau_f]}$ such that for all $d(\cdot) \in \mathcal{D}_{[\tau_i, \tau_f]}$, the system trajectory satisfies $x(\tau_f)\in R$ and $x(t) \in A^c$ for all $t \in [\tau_i, \tau_f] $:
\begin{multline}
    \label{ras_term}
    RAS_{\tau_f}(t) = 
    \left\{
        x_i \in \mathbb{R}^n | \exists \gamma(\cdot) \in \Gamma_{[t, \tau_f]},   \forall d(\cdot) \in \mathcal{D}_{[t, \tau_f]} 
    \right.\\
        (\phi(\tau_f; x_i, t, \gamma(\cdot), d(\cdot)) \in R) \quad \wedge \quad \\ (\forall \tau \in [t, \tau_f], \phi(\tau; x_i, t, \gamma(\cdot), d(\cdot)) \notin A)
    \left. 
    \right\}
\end{multline}

The cost function $J(\cdot)$ of the game now must take into account also the presence of obstacles along the trajectory, therefore we define it as:
\begin{multline}
    \label{j_ras_term}
    J(x, t, u(\cdot), d(\cdot)) =  
        \max
        \left\{
            g(\mathcal{X}(\tau_f)), 
            \max_{\tau \in [\tau_i, \tau_f]}  h(\mathcal{X}(\tau))
        \right\}
\end{multline}
Where: 
\[\mathcal{X}(\tau)=\phi(\tau; x, t, u(\cdot), d(\cdot))\]

Then the value function $V: \mathbb{R}^n \times [\tau_i, \tau_f] \rightarrow R$ is given by:
\[ V(x, t) = \inf_{\gamma(\cdot) \in \Gamma_{[t, \tau_f]}} \sup_{d(\cdot) \in \mathcal{D}_{[t, \tau_f]}} J(x, t, \gamma(\cdot), d(\cdot)) \]

The RAS is linked to the level set of the value function $V(\cdot)$ through the following proposition proved in \cite{reach_avoid_with_dist} :
\begin{prop}
    \label{prop_ras_term}
    $RAS_{\tau_f}(t) = \left\{x \in \mathbb{R}^n | V(x,t) \leq 0\right\}$
\end{prop}
We are finally ready to introduce the theorem that allows us to compute $V(\cdot)$ and then thanks to Prop.\ref{prop_ras_term}, to calculate $RAS_{term}$. The proof can be found in \cite{reach_avoid_with_dist}.
\begin{theorem}
    $V(\cdot)$ is the unique viscosity solution over $(x,t) \in \mathbb{R}^n \times [\tau_i, \tau_f]$ of the variational inequality:
    \[
        \max 
        \left\{
            h(x)-V(x,t), \frac{\partial V}{\partial t}(x,t)+H(x, t)
        \right\} = 0
    \]
    \[
        H(x, t) = \sup_{d \in D}\inf_{u \in U} \frac{\partial V}{\partial t}(x,t)f(x, u, d)
    \]
    with terminal condition:
    \[
        V(x,\tau_f)=\max\left\{g(x), h(x)\right\}
    \]
\end{theorem}

\subsection{Reach-avoid at any time}
The second type of reachability problem is similar to the previous one, however, in this case we are not interested to reach the target set exactly at $t=\tau_f$ but at any time $t \in [\tau_i, \tau_f]$, therefore, the RAS now contains the set of initial states $x_i$ from which the system trajectory can start and, using an optimal control input $u$, reaches the target set $R$ at some time $t$ without passing through the set $A$ until it hits $R$.
\begin{multline}
    \label{ras_t}
    \widetilde{RAS}_{t}(t) = 
    \left\{
        x_i \in \mathbb{R}^n | \exists \gamma(\cdot) \in \Gamma_{[t, \tau_f]},   \forall d(\cdot) \in \mathcal{D}_{[t, \tau_f]} 
    \right.\\
        \exists \tau_1 \in [t, \tau_f],
        (\phi(\tau_1; x_i, t, \gamma(\cdot), d(\cdot)) \in R) \quad \wedge \quad \\ (\forall \tau_2 \in [t, \tau_f], \phi(\tau_2; x_i, t, \gamma(\cdot), d(\cdot)) \notin A)
    \left. 
    \right\}
\end{multline}
For technical reasons related to the current differential game \cite{reach_avoid_with_dist} \cite{mitchell_time_dep_HJ}, it is necessary to define an augmented system dynamics in which Player 1 uses an augmented input $\widetilde{u}=[u, \overline{u}] \in U \times [0,1]$
\[
    \widetilde{f}(x, \widetilde{u}, d) = \overline{u}f(x,u,d)
\]
Assume $\widetilde{U}$, $\widetilde{\mathcal{U}}$, $\widetilde{\Gamma}$ defined similarly to the previous case, we denote the augmented system trajectory as $\widetilde{\phi}(\tau; x_i, t, \widetilde{u}(\cdot), d(\cdot))$. The value function is then similar to the previous case:
\begin{equation}
    \label{aug_v}
    \widetilde{V}(x,t)=
        \inf_{\widetilde{\gamma}(\cdot) \in \widetilde{\Gamma}_{[t, \tau_f]}} 
        \sup_{d(\cdot) \in \mathcal{D}_{[t, \tau_f]}} 
            \widetilde{J}(x, t, \widetilde{\gamma}(\cdot), d(\cdot)) 
\end{equation}
Where
\[
    \widetilde{J}(x, t, \widetilde{u}(\cdot), d(\cdot)) = 
        \max
        \left\{
            g(\widetilde{\mathcal{X}}(\tau_f)), 
            \max_{\tau \in [\tau_i, \tau_f]}  h(\widetilde{\mathcal{X}}(\tau))
        \right\}
\]
\[
    \widetilde{\mathcal{X}}(\tau)=\widetilde{\phi}(\tau; x, t, \widetilde{u}(\cdot), d(\cdot))
\]
Also in this case the RAS is linked to the value function $\widetilde{V}(x,t)$, proof in \cite{reach_avoid_with_dist}.
\begin{prop}
    \label{ras_v_any_time}
    For any $t \in [\tau_i, \tau_f]$: 
    \[
        \widetilde{RAS}_t(t)=
        \left\{
            x \in \mathbb{R}^n|\widetilde{V}(x,t) \leq 0
        \right\}
    \]
\end{prop}
Finally, the following theorem allows us to compute the RAS also in this case. Proof in \cite{reach_avoid_with_dist}.
\begin{theorem}
    $\widetilde{V}(\cdot)$ is the unique viscosity solution over $(x,t) \in \mathbb{R}^n \times [\tau_i, \tau_f]$ of the variational inequality:
    \[
        \max 
        \left\{
            h(x)-V(x,t), \frac{\partial \widetilde{V}}{\partial t}(x,t)+\min
            \left\{
                0, \widetilde{H}(x, t)
            \right\}
        \right\} = 0
    \]
    \[
        \widetilde{H}(x, t) = \sup_{d \in D}\inf_{u \in U} \frac{\partial \widetilde{V}}{\partial t}(x,t)f(x, u, d)
    \]
    with terminal condition:
    \[
        \widetilde{V}(x,\tau_f)=\max\left\{g(x), h(x)\right\}
    \]
\end{theorem}