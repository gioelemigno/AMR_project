Our case of study focouses on a three-wheel omni-directional mobile robot. Each wheel of the robot can move in each angular direction. The wheels are mounted simmetrycally with 120 degrees from each other and they are at a distance $D$  from the center of mass $C_{m}$. We can represent the robot in the space with a set of coordinates. First of all we call the coordinates of the Fixed GLobal System with $X_{w}$ and $Y_{w}$, so we can define the mobile robot fixed frame as $X_{m}$ and $Y_{m}$ (Those coordinates have their origin in the center of mass $C_{m}$).Then we can specify the direction angle of $X_{m}$ axis which we can call $\theta$. In addiction we can define two more angles called $\alpha$ and $\phi$: $\alpha$ describes the angle of the linear velocity of the robot $v_{r}$ w.r.t. the global system, while $\phi$ is the angle of the linear velocity of the robot $v_{r}$ w.r.t. the robot coordinate system. The linear velocity is related to the angular ones by the kinematic model w.r.t. the robot coordinates:
MODELLO

Where $v = [\dot{x}^{m}_{R},\dot{y}^{m}_{R},\omega]$ is the vector of robot velocities while $\dot{q} = [\dot{q}_{1},\dot{q}_{2},\dot{q}_{3}]$ is the vector of angular velocities of wheels multiplied by the wheel's radius.
The kinematic model w.r.t. the global coordinate system links the robot's velocities to the angular velocities.

MODELLO

In this equation we can see a tem multiplied by $d$, this term represents an introduced disturbance in the System. $\delta$ instead represents the orientation of the three wheels.As specified in the previous chapter the goal of the control system is to reach the Target set avoiding obstacles. We can define $\Phi_{G}\leq0$ as the disequaton to satisfy for the target Set and 

\begin{equation}
	\label{ode}
	\left\{
		\begin{array}{ll}
			\Phi_{A}(x)>0 \text{ if } x \in A \\
			\Phi_{A}(x)\leq 0 \text{ if } x \notin A
		\end{array}
	\right.
\end{equation}

After that we can definetly define the Constrained HJI partial differential equations with 
\begin{equation}
    \frac{\partial{\Phi}}{\partial{t}} + \min \left[ 0,H \left( x,\frac{\partial{\Phi}}{\partial{x}} \right) \right] = 0
\end{equation}
The above equation is subject to 
\begin{equation}
    \Phi(x,t)\geq - \Phi_{A}(x)
\end{equation}

The optimized Hemiltonian is here: EQUAZIONE LUNGA

Here we can define $u_{1}$, $u_{2}$ and $u_{3}\in[U_{1}, U_{2}]$. We can choose here $U_{1}=-U_{2}$ to get the optimal control inputs (PERCHÈ????)
\\
EQUAZIONI LUNGHISSIME
\\
Here we have not to check for all combinations of $u_{1}$ and $u_{2}$ and $u_{3}\in[U_{1},U_{2}]$ or this scenario would be very complex in terms of computation time

