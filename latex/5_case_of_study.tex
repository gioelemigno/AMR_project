In order to use HJ-RA to solve a reachability problem in the case of a robot, we have to follow several steps. The first one is the definition of the game dynamics, that in the case of wheeled mobile robot, is usually given by its kinematic model where control input is the wheels velocities. The second step is the definition and computation of the Hamiltonian, in our case the definition is (\ref{h_m}) since we assume $u$ wants to minimize and $d$ to maximize. In addition to that, the numerical method \cite{new_paper} able to compute $V^+(x,t)$ requires also computation of its gradient w.r.t. the vector $p$. The third step is the environment definition, namely definition of $g(x,s)$ and $h(x,s)$ that represent respectively the reach set $\mathcal{R}_s$ and the constraint set $\mathcal{K}_s$ at the time $s$. The final step is the numerical computation of $V^+(x,t)$ using the algorithm in \cite{new_paper}, to do that we can use two Matlab libraries: ToolboxLS \cite{LS} and helperOC \cite{brief_intro}.

Once computed $V^+(x,t)$, in a limited and discretized state-time space, we can use this value function to define a safe control law from each feasible initial states $x_0$ to the reach set.

\subsection{Definition Game Dynamics}
Our case of study focuses on a three-wheeled omnidirectional mobile robot which is a holonomic robot that has the ability to move simultaneously and independently in translation and rotation. The robot has three omni-wheels equally disposed at $120^\circ$, and have a distance $L$ from their center and the robot's center of mass $R \equiv O_m$.  

\begin{figure}[ht]
  \centering
  \includegraphics
      [width=0.35\textwidth]
      {figures/robot1.png}
  \caption{Three-wheeled omnidirectional mobile robot}
  \label{fig:robot_w}
\end{figure}

As shown in Fig. \ref{fig:robot_w}, there are two different reference frames: the first $RF_m \triangleq [O_m, X_m, Y_m]$ is a mobile frame placed at the robot's center of mass, the second one is the fixed world coordinate system $RF_w \triangleq [O_w, X_w, Y_w]$. Robot orientation w.r.t to $RF_w$ is given by the angle $\theta$. The angle $\delta$ is equal to $30^\circ$ and indicates the wheel orientation w.r.t. $RF_m$.

The kinematic model w.r.t. $RF_w$ is deduced as follows:

\begin{align}
    \dot{x} &= \begin{bmatrix}
           \frac{2}{3}\cos{(\theta + \delta)} & -\frac{2}{3}\cos{(\theta - \delta)} &  \frac{2}{3}\sin{\theta}   \\
           \frac{2}{3}\sin{(\theta + \delta)} & -\frac{2}{3}\sin{(\theta - \delta)} &  -\frac{2}{3}\cos{\theta} \\
           \frac{1}{3L} & \frac{1}{3L} & \frac{1}{3L}
         \end{bmatrix} \dot{q} 
  \label{eq:kin_model}
\end{align}
\begin{align*}
    &+ \begin{bmatrix}
         -sin{\theta} \\
         cos{\theta} \\
         0
        \end{bmatrix} d
\end{align*}

Where system state is given by $x = [^w x_R, ^w y_R, \theta]^T$. The second term represents an external disturbance given by a velocity with module $d$ and perpendicular to robot orientation $\theta$ (i.e. it lies on $Y_m$). The procedure to calculate (\ref{eq:kin_model}) is shown in the Appendix.

\begin{comment}
In this equation we can see a term multiplied by $d$, this term represents an introduced disturbance in the System. $\delta$ instead represents the orientation of the three wheels.So now we know that the reach-avoid set $RAS^\pm$ is described by the zero sublevel of the game's outcome $V^\pm(x,t)$. So given the model $f(x,u,d)$: 
\begin{equation*}
\begin{split}
f(\cdot) &= \\ \begin{pmatrix}
    \frac{2}{3}\cos{(\theta + \delta)}u_{1} - \frac{2}{3}\cos{(\theta - \delta)}u_{2} +\frac{2}{3}\sin{(\theta)}u_{3} - \sin{(\theta)d
    _{1}} \\
    \frac{2}{3}\sin{(\theta + \delta)}u_{1} - \frac{2}{3}\sin{(\theta - \delta)}u_{2} -\frac{2}{3}\cos{(\theta)}u_{3} - \cos{(\theta)}d
    _{2} \\
     \frac{1}{3L}u_{1} + \frac{1}{3L}u_{2} + \frac{1}{3L}u_{3}
\end{pmatrix}
\end{split}
\end{equation*}

The upper Hamiltonian is here:

\begin{equation}
    H^{*}(x,p) = \min_{u}\max_{d}(P_{1},P_{2},P_{3})f(x,u,d)
\end{equation}


So that $H^{*}$ becomes:
\begin{equation}
    \begin{split}
    H^{*} = \min_{u}\max_{d} &\begin{pmatrix}\frac{2}{3}\cos{(\theta + \delta)}P_{1} +  \frac{2}{3}\sin{(\theta + \delta)}P_{2} + \frac{1}{3L}P_{3}
    \end{pmatrix}u_{1} \\
    +  &\begin{pmatrix}-\frac{2}{3}\cos{(\theta -\delta)}P_{1} -\frac{2}{3}\sin{(\theta -\delta)}P_{2} + \frac{1}{3L}P_{3} \end{pmatrix}u_{2} \\
    +  &\begin{pmatrix}-\frac{2}{3}\sin{(\theta)}P_{1} -\frac{2}{3}\cos{(\theta)}P_{2} + \frac{1}{3L}P_{3} \end{pmatrix}u_{3} \\
    + &(-\sin{(\theta)}P_{1})d_{1}
    \\
    +&(\cos{(\theta)}P_{2})d_{2}
    \end{split}
\end{equation}


Here we can define $u_{1}$, $u_{2}$ and $u_{3}\in[U_{min}, U_{max}]$. We can choose here $U_{min}=-U_{max}$ to get the optimal control inputs so that:
    \begin{equation*}
    \begin{split}
        u^{*}_{1} &=\min_{u_{1}}\begin{pmatrix}\frac{2}{3}\cos{(\theta +  \delta)}P_{1} +  \frac{2}{3}\sin{(\theta + \delta)}P_{2} + \frac{1}{3L}P_{3}\end{pmatrix}u_{1} \\
         &=\min_{u_{1}}\gamma_{1}u_{1}
    \end{split}
    \end{equation*}
    \begin{equation}
    u^{*}_{1}=\begin{cases}
      U_{min}\quad \textrm{if} \quad \gamma_{1} \geq 0 \\
      U_{max}\quad \textrm{if} \quad \gamma_{1} < 0
    \end{cases}
\end{equation}
\\
    \begin{equation*}
    \begin{split}
        u^{*}_{2} &=\min_{u_{2}}\begin{pmatrix}-\frac{2}{3}\cos{(\theta -\delta)}P_{1} -\frac{2}{3}\sin{(\theta -\delta)}P_{2} + \frac{1}{3L}P_{3} \end{pmatrix}u_{2} \\
         &=\min_{u_{2}}\gamma_{2}u_{2}
        \end{split}
    \end{equation*}
    \begin{equation}
    u^{*}_{2}=\begin{cases}
      U_{min}\quad \textrm{if} \quad \gamma_{2} \geq 0 \\
      U_{max}\quad \textrm{if} \quad \gamma_{2} < 0
    \end{cases}
\end{equation}
%\\
    \begin{equation*}
    \begin{split}
        u^{*}_{3} &=\min_{u_{3}}\begin{pmatrix}-\frac{2}{3}\sin{(\theta)}P_{1} -\frac{2}{3}\cos{(\theta)}P_{2} + \frac{1}{3L}P_{3} \end{pmatrix}u_{3} \\
         &=\min_{u_{3}}\gamma_{3}u_{3}
        \end{split}
    \end{equation*}
    \begin{equation}
    u^{*}_{3}=\begin{cases}
      U_{min}\quad \textrm{if} \quad \gamma_{3} \geq 0 \\
      U_{max}\quad \textrm{if} \quad \gamma_{3} < 0
    \end{cases}
\end{equation}

\end{comment}
%\\
%\newpage

